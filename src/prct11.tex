\documentclass{beamer}
\usepackage[utf8]{inputenc}
\usepackage{graphicx}

\newtheorem{definicion}{Definición}
\newtheorem{ejemplo}{Ejemplo}

%%%%%%%%%%%%%%%%%%%%%%%%%%%%%%%%%%%%%%%%%%%%%%%%%%%%%%%%%%%%%%%%%%%%%%%%%%%%%%%
\title[Numero Pi]{Práctica 11}
\author[Laura Hernández Bethencourt]{Técnicas Experimentales}
\date[23-04-2014]{24 de abril de 2014}
%%%%%%%%%%%%%%%%%%%%%%%%%%%%%%%%%%%%%%%%%%%%%%%%%%%%%%%%%%%%%%%%%%%%%%%%%%%%%%%

\usetheme{Madrid}
%\usetheme{Antibes}
%\usetheme{tree}
%\usetheme{classic}

%%%%%%%%%%%%%%%%%%%%%%%%%%%%%%%%%%%%%%%%%%%%%%%%%%%%%%%%%%%%%%%%%%%%%%%%%%%%%%%
\begin{document}
  
%++++++++++++++++++++++++++++++++++++++++++++++++++++++++++++++++++++++++++++++  
\begin{frame}
  \titlepage
  \begin{small}
    \begin{center}
     Facultad de Matemáticas \\
     ULL
    \end{center}
  \end{small}

\end{frame}
%++++++++++++++++++++++++++++++++++++++++++++++++++++++++++++++++++++++++++++++  

%++++++++++++++++++++++++++++++++++++++++++++++++++++++++++++++++++++++++++++++  
\begin{frame}
  \frametitle{Índice}  
  \tableofcontents[pausesections]
\end{frame}
%++++++++++++++++++++++++++++++++++++++++++++++++++++++++++++++++++++++++++++++  

\section{Definición}

%++++++++++++++++++++++++++++++++++++++++++++++++++++++++++++++++++++++++++++++  
\begin{frame}

\frametitle{Definición}

\begin{definicion}
El número \alert{PI}~\cite{wiki} es la relación entre la longitud de una circunferencia y su diámetro, en geometría euclidiana. 

\end{definicion}

\end{frame}
%++++++++++++++++++++++++++++++++++++++++++++++++++++++++++++++++++++++++++++++  

\section{Fórmulas}

%++++++++++++++++++++++++++++++++++++++++++++++++++++++++++++++++++++++++++++++  
\begin{frame}

\frametitle{Fórmulas}

\begin{block}{Ejemplo}
  \begin{itemize}
  \item
  $\pi \approx \frac{2nl}{xt}.$
  \pause

  \item
  $i^i=\left(e^{i\pi /2}\right)^i=e^{i^2\pi /2}=e^{-\pi /2}=0.207879...$
  \pause

  \item
  $\int_{-\infty}^{\infty}e^{-x^2}dx=\sqrt{\pi}.$
  
  \pause

  \item
  $\pi \approx 3,1415926535 \; 8979323846 \; 2643383279 \; 5028841971 \; 6939937510$
  \pause

  \item
  $\pi \simeq \frac{377}{120} = 3{,}1416 \ldots$
  \pause
  
  \end{itemize}
\end{block}

\end{frame}
%++++++++++++++++++++++++++++++++++++++++++++++++++++++++++++++++++++++++++++++  

\section{Ejercicios}

\subsection{Subsección de prueba 1}
%++++++++++++++++++++++++++++++++++++++++++++++++++++++++++++++++++++++++++++++  
\begin{frame}
\frametitle{Diapositiva 5}

Texto de la diapositiva \alert{numero 5.}\cite{beamer}
\end{frame}
%++++++++++++++++++++++++++++++++++++++++++++++++++++++++++++++++++++++++++++++  

\subsection{Subsección de prueba 2}

%++++++++++++++++++++++++++++++++++++++++++++++++++++++++++++++++++++++++++++++  
\begin{frame}
\frametitle{Práctica}

\begin{definition}
  Definicióoooooooooon.
\end{definition}

\begin{example}
  \begin{itemize}
    \item <1-> Práctica \pause
    \item <2-> del \pause
    \item <3-> número \pause
    \item <4-> PI  
  \end{itemize}
\end{example}

\end{frame}
%++++++++++++++++++++++++++++++++++++++++++++++++++++++++++++++++++++++++++++++  

\section{Bibliografía}
%++++++++++++++++++++++++++++++++++++++++++++++++++++++++++++++++++++++++++++++  
\begin{frame}
  \frametitle{Bibliografía}

  \begin{thebibliography}{10}

    \beamertemplatebookbibitems
    \bibitem[Wikipedia]{wiki}  
    Wikipedia.
    {\small $http://es.wikipedia.org$}
    
    \beamertemplatebookbibitems
    \bibitem[Beamer]{beamer} 
    Tutorial beamer. {\small $http://campusvirtual.ull.es/1314/pluginfile.php/197674/mod_resource/content/1/beameruserguide.pdf$}

  \end{thebibliography}
\end{frame}

%++++++++++++++++++++++++++++++++++++++++++++++++++++++++++++++++++++++++++++++  
\end{document}
